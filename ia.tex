\documentclass[12pt]{article}
\usepackage[margin=1in]{geometry}
\usepackage{siunitx}
\usepackage[version=4]{mhchem}
\usepackage{xltabular}
\usepackage{ragged2e}
\usepackage[font=small, skip=3pt]{caption}
\usepackage{float}
\usepackage{amsmath}


\DeclareSIUnit{\mpl}{\mol\per\litre}
\DeclareSIUnit{\gpl}{\gram\per\mol}
\DeclareSIUnit{\ml}{\milli\litre}

\sisetup{space-before-unit = true, free-standing-units = true}

\begin{document}


\section*{Purpose}
Investigating the relationship between the time tap water is boiled and the amount of chlorine concentration remaining.

\section*{Materials}

<?#
I ended up using 2 mL total of 1 mol/L K2CrO4
This produces 0.01M

I remember using 0.01M of AgNO3 as well
?>

\begin{itemize}
	\item Burette
	\item 0.1\mpl~Silver Nitrate Solution
	\item 0.25\mpl~Potassium Chromate Indicator Solution
	\item 250\ml~Erlenmyler Flask
	\item Hot Plate
\end{itemize}

\section*{Procedure}

\begin{enumerate}
	\item A silver nitrate solution (\ce{AgNO3(aq)}) was created by measuring out 0.0817\g of solid \ce{AgNO3} and diluting it with 50\ml of deionized water.
	\item A dilute solution of 0.01\mpl \ce{K2CrO4} was created by measuring out 2\ml of the 0.1\mpl stock solution and diluting it with 10\ml of deionized water.
	\item The \ce{AgNO3} solution was poured into a burette.
	\item 10\ml of tap water was transported into an 100\ml Erlenmyler flask using a 10\ml pipette.
	\item Approximately 1\ml of the 0.01\mpl \ce{K2CrO4} solution was added to the 10\ml of tap water using a dropper.
	\item The initial volume of \ce{AgNO3} solution in the burette was noted down. The tap water was titrated with the \ce{AgNO3} solution until the solution turned reddish-brown (see Figure x). The final amount of solution in the burette was noted down.
	\item Steps 4 to 6 were repeated for 2-3 more trials.
	\item Steps 4 to 7 were repeated for deionized water and water boiled for 3, 6, 9, and 20 minutes using a hot plate.
\end{enumerate}

\section*{Data}

% \begin{noindent}
<?
/*
[0]: # of minutes boiled (or 'deionized' if the water was deionized)
[1]: mL of AgNO3 needed
[2]: mL of H2O used
*/

const rawData = [
	[0, 11.45 - 10.15, 10],
	[0, 17.45 - 16.2, 10],
	[0, 18.7 - 17.45, 10],
	[3, 5.6 - 4.3, 10],
	[3, 14.3 - 13, 10],
	[3, 13 - 11.75, 10],
	[6, 20.15 - 18.7, 10],
	[6, 30.8 - 29.4, 10],
	[6, 7 - 5.6, 10],
	[9, 8.8 - 7, 10],
	[9, 16.2 - 14.3, 10],
	[9, 10.15 - 8.65, 10],
	[20, 26.4 - 20.15, 10],
	[20, 29.4 - 26.4, 5],
	['deionized', 11.75 - 11.45, 10],
	['deionized', 31.15 - 30.8, 10],
	['deionized', 31.6 - 31.15, 10],
];

const headers = ['minutesBoiled', 'titrantAmount', 'waterAmount'];

const data = rawData.map(row => Object.fromEntries(R.zip(headers, row)));
?>
% \end{noindent}

\begin{table}[H]
	\caption{TODO}
	\def\arraystretch{1.5}
	\begin{tabularx}{\linewidth}{|
			>{\RaggedRight}X|
			>{\RaggedRight}X|
			>{\RaggedRight}X|
			>{\RaggedRight}X|
		}
		\hline
		\textbf{Trial Number}         &
		\textbf{Time Boiled /\minute} &
		\textbf{Titrant Amount /\ml}  &
		\textbf{Amount of Water /\ml}
		\\\hline
		% \begin{noindent}
		<? for (const [rowIndex, row] of data.entries()) { ?>
			Trial <?= rowIndex + 1 ?>
			& <?= row.minutesBoiled ?>
			& <?= sigfig(row.titrantAmount, 3) ?>
			& <?= row.waterAmount ?>
			\\\hline
		<? } ?>
		% \end{noindent}
	\end{tabularx}
\end{table}

\section*{Analysis}

To determine the amount of chlorine ions in the water, the amount of water and the amount of titrant is used.

The equation of the reaction can be represented as follows:

\centerline{\ce{AgNO3_{(aq)} + Cl^{-}_{(aq)} -> AgCl_{(s)} + NO3^{-}_{(aq)}}}

\section*{Calculations}

% TODO: add reference
Because the concentration of Chloride ions in tap water is very trace (maximum of \SI{4}{\mg\per\litre}), I need to dilute the Silver Nitrate Solution.

To calculate how much I should dilute the \ce{Ag(NO)3} solution by, I calculated the theoretical volume of a 0.1\mpl \ce{Ag(NO)3} solution that would be needed to titrate 100\ml of tap water:

\begin{align*}
	100\ml~\ce{H2O} \times \frac{4\mg}{L} MM_{\ce{Cl}} & = 35.45\gpl
	\\
	n_{Cl}                                             & = \frac{0.4\mg}{35.45}
\end{align*}

\end{document}
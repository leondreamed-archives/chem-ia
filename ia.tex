\documentclass[12pt]{article}
\usepackage[margin=1in]{geometry}
\usepackage{siunitx}
\usepackage[version=4]{mhchem}
\usepackage{xltabular}
\usepackage{ragged2e}
\usepackage[font=small, skip=3pt]{caption}
\usepackage{float}
\usepackage{amsmath}

\DeclareSIUnit{\mpl}{\mol\per\litre}
\DeclareSIUnit{\gpl}{\gram\per\mol}
\DeclareSIUnit{\ml}{\milli\litre}

\sisetup{space-before-unit = true, free-standing-units = true}

\begin{document}

% \begin{noindent}
<?
/*
[0]: # of minutes boiled (or 'deionized' if the water was deionized)
[1]: mL of AgNO3 needed
[2]: mL of H2O used
*/

const rawData = [
	[0, 11.45 - 10.15, 10],
	[0, 17.45 - 16.2, 10],
	[0, 18.7 - 17.45, 10],
	[3, 14.3 - 13, 10],
	[3, 13 - 11.75, 10],
	[6, 20.15 - 18.7, 10],
	[6, 30.8 - 29.4, 10],
	[9, 16.2 - 14.3, 10],
	[9, 10.15 - 8.65, 10],
	[20, 26.4 - 20.15, 10],
	[20, 29.4 - 26.4, 5],
	['deionized', 11.75 - 11.45, 10],
	['deionized', 31.6 - 31.15],
];

const headers = ['minutesBoiled', 'titrantAmount', 'waterAmount'];

const data = rawData.map(row => R.zip(row, headers));
?>
% \end{noindent}

<?= escapeLatex(JSON.stringify(data)) ?>

\section*{Purpose}
Investigating the relationship between the time tap water is boiled and the amount of chlorine concentration remaining.

\section*{Materials}

\begin{itemize}
	\item Burette
	\item 0.1\mpl~Silver Nitrate Solution
	\item 0.25\mpl~Potassium Chromate Indicator Solution
	\item 250\ml~Erlenmyler Flask
\end{itemize}

\section*{Procedure}

% TODO: add reference
Because the concentration of Chloride ions in tap water is very trace (maximum of \SI{4}{\mg\per\litre}), I need to dilute the Silver Nitrate Solution.

To calculate how much I should dilute the \ce{Ag(NO)3} solution by, I calculated the theoretical volume of a 0.1\mpl \ce{Ag(NO)3} solution that would be needed to titrate 100\ml of tap water:

\begin{align*}
	100\ml~\ce{H2O} \times \frac{4\mg}{L} MM_{\ce{Cl}} & = 35.45\gpl
	\\
	n_{Cl}                                             & = \frac{0.4\mg}{35.45}
\end{align*}

\end{document}
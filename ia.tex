\documentclass[12pt]{article}
\usepackage[margin=1in]{geometry}
\usepackage{siunitx}
\usepackage[version=4]{mhchem}
\usepackage{xltabular}
\usepackage{ragged2e}
\usepackage[font=small, skip=3pt]{caption}
\usepackage{float}
\usepackage{amsmath}

\DeclareSIUnit{\mpl}{\mol\per\litre}
\DeclareSIUnit{\gpl}{\gram\per\mol}
\DeclareSIUnit{\ml}{\milli\litre}

\sisetup{space-before-unit = true, free-standing-units = true}

\begin{document}

\section*{Purpose}
Investigating the relationship between the time tap water is boiled and the amount of chlorine concentration remaining.

\section*{Materials}

\begin{itemize}
	\item Burette
	\item 0.1\mpl Silver Nitrate Solution
	\item 0.25\mpl Potassium Chromate Indicator Solution
	\item 250\ml Erlenmyler Flask
\end{itemize}

\section*{Procedure}

% TODO: add reference
Because the concentration of Chloride ions in tap water is very trace (maximum of 4\mg\per\litre), I need to dilute the Silver Nitrate Solution.

To calculate how much I should dilute the \ce{Ag(NO)3} solution by, I calculated the theoretical volume of a 0.1\mpl \ce{Ag(NO)3} solution that would be needed to titrate 100\ml of tap water:

\begin{align*}
100\ml \ce{H2O} * \frac{4\mg}{L}
MM_\ce{Cl} = 35.45\gpl

n_{Cl} = \frac{0.4mg}{35.45}
\end{align*}



\end{document}